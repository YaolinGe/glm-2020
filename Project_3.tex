% Options for packages loaded elsewhere
\PassOptionsToPackage{unicode}{hyperref}
\PassOptionsToPackage{hyphens}{url}
%
\documentclass[
]{article}
\usepackage{lmodern}
\usepackage{amssymb,amsmath}
\usepackage{ifxetex,ifluatex}
\ifnum 0\ifxetex 1\fi\ifluatex 1\fi=0 % if pdftex
  \usepackage[T1]{fontenc}
  \usepackage[utf8]{inputenc}
  \usepackage{textcomp} % provide euro and other symbols
\else % if luatex or xetex
  \usepackage{unicode-math}
  \defaultfontfeatures{Scale=MatchLowercase}
  \defaultfontfeatures[\rmfamily]{Ligatures=TeX,Scale=1}
\fi
% Use upquote if available, for straight quotes in verbatim environments
\IfFileExists{upquote.sty}{\usepackage{upquote}}{}
\IfFileExists{microtype.sty}{% use microtype if available
  \usepackage[]{microtype}
  \UseMicrotypeSet[protrusion]{basicmath} % disable protrusion for tt fonts
}{}
\makeatletter
\@ifundefined{KOMAClassName}{% if non-KOMA class
  \IfFileExists{parskip.sty}{%
    \usepackage{parskip}
  }{% else
    \setlength{\parindent}{0pt}
    \setlength{\parskip}{6pt plus 2pt minus 1pt}}
}{% if KOMA class
  \KOMAoptions{parskip=half}}
\makeatother
\usepackage{xcolor}
\IfFileExists{xurl.sty}{\usepackage{xurl}}{} % add URL line breaks if available
\IfFileExists{bookmark.sty}{\usepackage{bookmark}}{\usepackage{hyperref}}
\hypersetup{
  pdftitle={Project 3 - TMA4315},
  pdfauthor={\textbackslash overset\{\textbackslash mathrm\{martin.o.berild@ntnu.no\}\}\{10014\} \textbackslash overset\{\textbackslash mathrm\{yaolin.ge@ntnu.no\}\}\{10026\}},
  hidelinks,
  pdfcreator={LaTeX via pandoc}}
\urlstyle{same} % disable monospaced font for URLs
\usepackage[margin=1in]{geometry}
\usepackage{color}
\usepackage{fancyvrb}
\newcommand{\VerbBar}{|}
\newcommand{\VERB}{\Verb[commandchars=\\\{\}]}
\DefineVerbatimEnvironment{Highlighting}{Verbatim}{commandchars=\\\{\}}
% Add ',fontsize=\small' for more characters per line
\usepackage{framed}
\definecolor{shadecolor}{RGB}{248,248,248}
\newenvironment{Shaded}{\begin{snugshade}}{\end{snugshade}}
\newcommand{\AlertTok}[1]{\textcolor[rgb]{0.94,0.16,0.16}{#1}}
\newcommand{\AnnotationTok}[1]{\textcolor[rgb]{0.56,0.35,0.01}{\textbf{\textit{#1}}}}
\newcommand{\AttributeTok}[1]{\textcolor[rgb]{0.77,0.63,0.00}{#1}}
\newcommand{\BaseNTok}[1]{\textcolor[rgb]{0.00,0.00,0.81}{#1}}
\newcommand{\BuiltInTok}[1]{#1}
\newcommand{\CharTok}[1]{\textcolor[rgb]{0.31,0.60,0.02}{#1}}
\newcommand{\CommentTok}[1]{\textcolor[rgb]{0.56,0.35,0.01}{\textit{#1}}}
\newcommand{\CommentVarTok}[1]{\textcolor[rgb]{0.56,0.35,0.01}{\textbf{\textit{#1}}}}
\newcommand{\ConstantTok}[1]{\textcolor[rgb]{0.00,0.00,0.00}{#1}}
\newcommand{\ControlFlowTok}[1]{\textcolor[rgb]{0.13,0.29,0.53}{\textbf{#1}}}
\newcommand{\DataTypeTok}[1]{\textcolor[rgb]{0.13,0.29,0.53}{#1}}
\newcommand{\DecValTok}[1]{\textcolor[rgb]{0.00,0.00,0.81}{#1}}
\newcommand{\DocumentationTok}[1]{\textcolor[rgb]{0.56,0.35,0.01}{\textbf{\textit{#1}}}}
\newcommand{\ErrorTok}[1]{\textcolor[rgb]{0.64,0.00,0.00}{\textbf{#1}}}
\newcommand{\ExtensionTok}[1]{#1}
\newcommand{\FloatTok}[1]{\textcolor[rgb]{0.00,0.00,0.81}{#1}}
\newcommand{\FunctionTok}[1]{\textcolor[rgb]{0.00,0.00,0.00}{#1}}
\newcommand{\ImportTok}[1]{#1}
\newcommand{\InformationTok}[1]{\textcolor[rgb]{0.56,0.35,0.01}{\textbf{\textit{#1}}}}
\newcommand{\KeywordTok}[1]{\textcolor[rgb]{0.13,0.29,0.53}{\textbf{#1}}}
\newcommand{\NormalTok}[1]{#1}
\newcommand{\OperatorTok}[1]{\textcolor[rgb]{0.81,0.36,0.00}{\textbf{#1}}}
\newcommand{\OtherTok}[1]{\textcolor[rgb]{0.56,0.35,0.01}{#1}}
\newcommand{\PreprocessorTok}[1]{\textcolor[rgb]{0.56,0.35,0.01}{\textit{#1}}}
\newcommand{\RegionMarkerTok}[1]{#1}
\newcommand{\SpecialCharTok}[1]{\textcolor[rgb]{0.00,0.00,0.00}{#1}}
\newcommand{\SpecialStringTok}[1]{\textcolor[rgb]{0.31,0.60,0.02}{#1}}
\newcommand{\StringTok}[1]{\textcolor[rgb]{0.31,0.60,0.02}{#1}}
\newcommand{\VariableTok}[1]{\textcolor[rgb]{0.00,0.00,0.00}{#1}}
\newcommand{\VerbatimStringTok}[1]{\textcolor[rgb]{0.31,0.60,0.02}{#1}}
\newcommand{\WarningTok}[1]{\textcolor[rgb]{0.56,0.35,0.01}{\textbf{\textit{#1}}}}
\usepackage{graphicx,grffile}
\makeatletter
\def\maxwidth{\ifdim\Gin@nat@width>\linewidth\linewidth\else\Gin@nat@width\fi}
\def\maxheight{\ifdim\Gin@nat@height>\textheight\textheight\else\Gin@nat@height\fi}
\makeatother
% Scale images if necessary, so that they will not overflow the page
% margins by default, and it is still possible to overwrite the defaults
% using explicit options in \includegraphics[width, height, ...]{}
\setkeys{Gin}{width=\maxwidth,height=\maxheight,keepaspectratio}
% Set default figure placement to htbp
\makeatletter
\def\fps@figure{htbp}
\makeatother
\setlength{\emergencystretch}{3em} % prevent overfull lines
\providecommand{\tightlist}{%
  \setlength{\itemsep}{0pt}\setlength{\parskip}{0pt}}
\setcounter{secnumdepth}{-\maxdimen} % remove section numbering

\title{Project 3 - TMA4315}
\author{\(\overset{\mathrm{martin.o.berild@ntnu.no}}{10014}\)
\and \(\overset{\mathrm{yaolin.ge@ntnu.no}}{10026}\)}
\date{\today}

\begin{document}
\maketitle

\hypertarget{problem-1}{%
\subsection{Problem 1}\label{problem-1}}

\hypertarget{a}{%
\paragraph{a)}\label{a}}

It can be shown as Assume data \((y_{ij}, x_{ij}^T)\) for
\(i = 1, 2, ..., m, j = 1, 2, ..., n_i\)

\[
y_{ij} = x_{ij}^T\beta + u_{ij}^T\gamma_i + \epsilon_{ij}
\] By applying the random intercept slope model

\[
y_{ij} = \begin{bmatrix} 1 & x_{ij}\end{bmatrix} \begin{bmatrix} \beta_0 \\ \beta_1\end{bmatrix} + \begin{bmatrix} 1 & x_{ij}\end{bmatrix} \begin{bmatrix} \gamma_{0, i} \\ \gamma_{1, i}\end{bmatrix} + \epsilon_{ij}
\] Letting

\[
y_i = \begin{bmatrix} y_{i1} \\ y_{i2} \\ $\vdots$ \\ y_{in_i} \end{bmatrix}, \ \ x_i = \begin{bmatrix} x_{i1}^T \\ x_{i2}^T \\ \vdots \\ x_{in_i}^T \end{bmatrix}, \ \ U_i = \begin{bmatrix} u_{i1}^T \\ u_{i2}^T \\ \vdots \\ u_{in_i}^T \end{bmatrix}, \ \ \epsilon_i = \begin{bmatrix} \epsilon_{i1} \\ \epsilon_{i2} \\ $\vdots$ \\ \epsilon_{in_i} \end{bmatrix}
\]

The the model is (for cluster \(i = 1, 2, \cdots, m\)): \[
y_i = X_i\beta + U_i\gamma_i + \epsilon_i, \ \ \gamma_i \sim \mathcal{N(0, Q)}
\]

The marginal model is then

\[
y_i \sim \mathcal{N}(X_i \beta,\ \  U_iQU_I^T + \sigma^2I_{n_i})
\]

Conditional model is \$\$

y\_i\textbar{}\gamma\_i \sim \mathcal{N}(X\_i\beta + U\_i\gamma\emph{i,
~~\sigma\^{}2I}\{n\_i\}) \$\$

We are interested in computing the maximum likelihood and restricted
maximum likelihood estimates of the parameters in the linear mixed model

\[
y_{ij} = \beta_0 + \beta_1x_{ij} + \gamma_{0,i} + \gamma_{1,i}x_{ij} + \epsilon_{ij},
\] where \(\boldsymbol{\gamma}_i = (\gamma_{0,i},\gamma_{1,i})\) are iid
binomially distributed with zero mean and covariance matrix

\[
\begin{bmatrix}
\tau_0^2 & \tau_{01} \\
\tau_{01} & \tau_{1}^2
\end{bmatrix},
\] and \(\epsilon_{ij}\) are idd Guassian distributed with zero mean and
variance \(\sigma^2\) for \(i = 1,\dots,m\) and \(j=1,\dots,n\). Thus,
we are interested in obtaining the estimates of
\((\beta_0,\beta_1,\tau_0^2,\tau_1^2,\tau_{01},\sigma^2)\).

\hypertarget{problem-2}{%
\subsection{Problem 2}\label{problem-2}}

Next, we are interested in modelling the 2018 results of the Norwegian
elite football league using a generalized linear mixed model. First we
load the data and display the contents.

\begin{Shaded}
\begin{Highlighting}[]
\NormalTok{long <-}\StringTok{ }\KeywordTok{read.csv}\NormalTok{(}\StringTok{"https://www.math.ntnu.no/emner/TMA4315/2020h/eliteserie.csv"}\NormalTok{, }\DataTypeTok{colClasses =} \KeywordTok{c}\NormalTok{(}\StringTok{"factor"}\NormalTok{,}\StringTok{"factor"}\NormalTok{,}\StringTok{"factor"}\NormalTok{,}\StringTok{"numeric"}\NormalTok{))}
\KeywordTok{head}\NormalTok{(long)}
\end{Highlighting}
\end{Shaded}

\begin{verbatim}
##               attack            defence home goals
## 1              Molde Sandefjord_Fotball  yes     5
## 2 Sandefjord_Fotball              Molde   no     0
## 3      Stroemsgodset            Stabaek  yes     2
## 4            Stabaek      Stroemsgodset   no     2
## 5                Odd          Haugesund  yes     1
## 6          Haugesund                Odd   no     2
\end{verbatim}

\end{document}
